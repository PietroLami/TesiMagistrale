%%%%%%%%%%%%%%%%%%%%%%%%%%%%%%%%%%%%%%%%%
% Beamer Presentation
% LaTeX Template
% Version 1.0 (10/11/12)
%
% This template has been downloaded from:
% http://www.LaTeXTemplates.com
%
% License:
% CC BY-NC-SA 3.0 (http://creativecommons.org/licenses/by-nc-sa/3.0/)
%
%%%%%%%%%%%%%%%%%%%%%%%%%%%%%%%%%%%%%%%%%

%----------------------------------------------------------------------------------------
%	PACKAGES AND THEMES
%----------------------------------------------------------------------------------------

\documentclass{beamer}

\mode<presentation> {

% The Beamer class comes with a number of default slide themes
% which change the colors and layouts of slides. Below this is a list
% of all the themes, uncomment each in turn to see what they look like.

%\usetheme{default}
%\usetheme{AnnArbor}
%\usetheme{Antibes}
%\usetheme{Bergen}
%\usetheme{Berkeley}
%\usetheme{Berlin}
%\usetheme{Boadilla}
%\usetheme{CambridgeUS}
%\usetheme{Copenhagen}
%\usetheme{Darmstadt}
%\usetheme{Dresden}
%\usetheme{Frankfurt}
%\usetheme{Goettingen}
%\usetheme{Hannover}
%\usetheme{Ilmenau}
%\usetheme{JuanLesPins}
%\usetheme{Luebeck}
%\usetheme{Madrid}
%\usetheme{Malmoe}
%\usetheme{Marburg}
%\usetheme{Montpellier}
%\usetheme{PaloAlto}
%\usetheme{Pittsburgh}
%\usetheme{Rochester}
%\usetheme{Singapore}
%\usetheme{Szeged}
\usetheme{Warsaw}

% As well as themes, the Beamer class has a number of color themes
% for any slide theme. Uncomment each of these in turn to see how it
% changes the colors of your current slide theme.

%\usecolortheme{albatross}
%\usecolortheme{beaver}
%\usecolortheme{beetle}
%\usecolortheme{crane}
%\usecolortheme{dolphin}
%\usecolortheme{dove}
%\usecolortheme{fly}
%\usecolortheme{lily}
%\usecolortheme{orchid}
%\usecolortheme{rose}
%\usecolortheme{seagull}
%\usecolortheme{seahorse}
%\usecolortheme{whale}
%\usecolortheme{wolverine}

%\setbeamertemplate{footline} % To remove the footer line in all slides uncomment this line
%\setbeamertemplate{footline}[page number] % To replace the footer line in all slides with a simple slide count uncomment this line

%\setbeamertemplate{navigation symbols}{} % To remove the navigation symbols from the bottom of all slides uncomment this line
}

\usepackage{graphicx} % Allows including images
\usepackage{booktabs} % Allows the use of \toprule, \midrule and \bottomrule in tables

%----------------------------------------------------------------------------------------
%	TITLE PAGE
%----------------------------------------------------------------------------------------

\title[]{Estensione di un debugger reversibile per Erlang con feature imperative} % The short title appears at the bottom of every slide, the full title is only on the title page

\author{} % Your name
\institute[UniBo] % Your institution as it will appear on the bottom of every slide, may be shorthand to save space
{
Universit\`{a} di Bologna \\ % Your institution for the title page
\medskip \medskip
SCUOLA DI SCIENZE\\
Corso di Laurea Magistrale in Informatica\\
\medskip \medskip
Relatore: \quad  Prof. Claudio Sacerdoti Coen \\ 
Presentata da: \qquad \qquad \qquad Pietro Lami\\
\medskip \medskip
II Sessione\\
Anno Accademico 2019-2020
}
\date{} % Date, can be changed to a custom date

\begin{document}


\begin{frame}
\titlepage % Print the title page as the first slide

\end{frame}

%\begin{frame}
%\frametitle{Introduzione} 
%Operational Transformation e Change Tracking sono due tecnologie usate nei sistemi di editing collaborativo real-time, software che permettono agli utenti di modificare lo stesso documento da macchine diverse nello stesso momento 
%\end{frame}

%----------------------------------------------------------------------------------------
%	PRESENTATION SLIDES
%----------------------------------------------------------------------------------------


%------------------------------------------------

\begin{frame}
\frametitle{Titolo}
\begin{block}{Definizione:}
 definizione
\begin{center}centro\end{center}
\end{block}

\end{frame}


%------------------------------------------------

\begin{frame}
\frametitle{Titolo 2}



\begin{itemize}
\item elenco   
\item puntato
\end{itemize}
\end{frame}


%------------------------------------------------

\begin{frame}
\frametitle{Titolo 3}
 \textcolor{blue}{Funzionamento:}\\~
\begin{columns}[c]
\column{.5\textwidth}
colonna 1 
\column{.5\textwidth}
\begin{itemize}
\item colonna 2.
\end{itemize}
\end{columns}
\end{frame}

%------------------------------------------------

\begin{frame}
\Huge{\centerline{Vi ringrazio per l'attenzione}}
\end{frame}

%----------------------------------------------------------------------------------------

\end{document} 